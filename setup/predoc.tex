\edef\hdrtitle{\thetitle}

\pagestyle{fancy}
\fancyhf{}
% The fancyhead commands put things in the header. The square brackets is which
% side and which pages (odd or even). So LO is left side of odd pages, and RE is
% right side of even pages.
% \fancyhead[LE,RO]{\hdrtitle}
% \fancyhead[RE,LO]{\leftmark}
\fancyhead[LO]{\hdrtitle}
% This is the chapter title.
\fancyhead[RE]{\leftmark}
% Same for the footer, this puts the page number in the centre for Even and Odd
% pages.
\fancyfoot[CE,CO]{\thepage}

% Override some settings for chapter pages - keep the header basically as-is but remove the \leftmark
\fancypagestyle{chap-fancy}{%
  \fancyhf{}
  %\fancyhead[LE,RO]{\hdrtitle}
  \fancyhead[LO]{\hdrtitle}
  \fancyfoot[CE,CO]{\thepage}
}
\patchcmd{\chapter}{\thispagestyle{chapter}}{\thispagestyle{chap-fancy}}{}{}

% This doesn't really impact anything, but you should include it because otherwise
% overleaf's PDF viewer gets janky. I don't know why that's the case, just trust
% me.
\hypersetup{
    colorlinks=false,
    linkcolor=blue,
    filecolor=magenta,
    urlcolor=cyan,
    pdftitle={Overleaf Example},
    pdfpagemode=FullScreen,
}

\newcommand{\bhref}[2]{\href{#1}{\bfseries#2}}

\newcommand{\bhyperref}[2]{\textbf{\hyperref[#1]{#2}}}

\newcommand{\bpageref}[1]{\textbf{\hyperref[#1]{page \pageref{#1}}}}

\newcommand{\corereference}{\textbf{\textsuperscript{T!}}}

% This sets the width between paragraphs, and the indent for the first line
\parskip=4pt
\parindent=0pt

% This sets the spacings around itemize lists and enumerated lists
\setlist[itemize]{leftmargin=1.5em,itemsep=0pt,topsep=0pt,parsep=4pt}
\setlist[enumerate]{leftmargin=2em,itemsep=0pt,topsep=0pt,parsep=4pt}

% center environment, with smaller parskips and separators
\newenvironment{tightcenter}{%
  \setlength\topsep{0pt}
  \setlength\parskip{4pt}
  \setlength\partopsep{0pt}
  \begin{center}
}{%
  \end{center}
}

% Informational "quote" style environment
\newmdenv[
    topline=false,
    bottomline=false,
    rightline=false,
    skipabove=\topsep,
    skipbelow=\topsep,
    linecolor=gray
]{informational}

% These next few blocks define the formatting for the chapter and section
% headings. Play around and modify things if you want, or leave them as is.
% More details on this in the https://ctan.org/pkg/titlesec titlesec package
% and overleaf's documentation https://www.overleaf.com/learn/latex/Sections_and_chapters#Customize_chapters_and_sections
\titleformat{\chapter}[hang]
{\scshape\LARGE} %Format
{\thechapter} %Label
{0.5em} %sec
{\vspace{-10mm}} %before-code
{} %after-code
\titlespacing{\chapter}{0pt}{-2pt}{0pt}

\titleformat{\section}[hang]
{\bfseries\Large}
{\thechapter}
{0.5em}
{}
{}
\titlespacing{\section}{0pt}{0pt}{0pt}

\titleformat{\subsection}[hang]
{\bfseries\large}
{\thechapter}
{0.5em}
{}
{}
\titlespacing{\subsection}{0pt}{0pt}{0pt}

\titleformat{\subsubsection}[hang]
{\bfseries}
{\thechapter}
{0.5em}
{}
{}
\titlespacing{\subsubsection}{0pt}{0pt}{0pt}

\titleformat{\paragraph}[runin]
{\bfseries}
{\thechapter}
{0.5em}
{}
{}
\titlespacing{\paragraph}{0pt}{0pt}{4pt}

% This turns off section numbering
\setcounter{secnumdepth}{-2}
% This sets the deepest section to include in the ToC.
\settocdepth{subsection}

\let\cleardoublepage\clearpage

% This is the command for a full-page title picture
\newcommand{\fullpagetitlepicture}[1]{
\AddToShipoutPictureBG*{\includegraphics[width=\paperwidth,height=\paperheight]{#1}}
\begin{titlingpage}
\begin{center}
\end{center}
\end{titlingpage}
}

% This command reformats the toc to be (IMO) "nicer". I can't exactly remember how
% or why this does what it does. But it does something. You can remove it and see
% what it looks like without it.
\makeatletter
\renewcommand\tableofcontents{%
 \@starttoc{toc}%
}
\makeatother

% Copyrightpage environment, puts text at bottom of page, with optional
% logo stamp.
\newenvironment{copyrightpage}[1]
{
    \vspace*{\fill}
    \thispagestyle{empty}
    \begin{center}
        \ifthenelse{\equal{#1}{}}{}{\includegraphics[width=20mm]{#1}}
    \end{center}
    
    \vspace*{\fill}
}
{
    \setcounter{page}{0}
    \pagenumbering{gobble}
    \clearpage
}

% nice 2 column table. No real reason to use this over nice2longtable, which
% works over multiple pages.
\newenvironment{nice2tabular}[3][1.4]
{
    \renewcommand{\arraystretch}{#1}
    \begin{tabular}{>{\raggedleft\arraybackslash}p{#2\linewidth};{1pt/3pt}>{\arraybackslash}p{#3\linewidth}}
}
{
    \end{tabular}
}

% nice multipage 2 column table. No real reason to use this over nice2longtable, which
% works over multiple pages.
\newenvironment{nice2longtable}[2]
{
    \renewcommand{\arraystretch}{1.4}
    \begin{longtable}{>{\raggedleft\arraybackslash}p{#1\linewidth};{1pt/3pt}>{\arraybackslash}p{#2\linewidth}}
}
{
    \end{longtable}
}

% Starts the appendix, with a full-page spacer just reading "Appendix", and then
% sets the header title to "Appendix"
\newcommand{\niceappendix}
{
    \clearpage
    \appendix

    \pagestyle{fancy}
    \fancyhf{}
    \fancyhead[LE,RO]{Appendix}
    \fancyhead[RE,LO]{\leftmark}
    \fancyfoot[CE,CO]{\thepage}

    \fancypagestyle{chap-fancy}{%
      \fancyhf{}
      \fancyhead[LE,RO]{Appendix}
      \fancyfoot[CE,CO]{\thepage}
    }
    \patchcmd{\chapter}{\thispagestyle{chapter}}{\thispagestyle{chap-fancy}}{}{}

    \thispagestyle{empty}
    \begin{vplace}
    \begin{center}
    \Huge\scshape Appendix    
    \end{center}
    \end{vplace}
    \clearpage
}

% The mientables!
\newenvironment{mientable}
{
    \begin{hyphenrules}{nohyphenation}
    \begin{tabular}{|>{\raggedright\arraybackslash}p{.06\linewidth}|>{\raggedright\arraybackslash}p{.76\linewidth}|}
    \hline
    \rowcolor[HTML]{000000}
    \multicolumn{2}{|c|}{{\color[HTML]{FFFFFF} Mien}}      \\ \hline
}
{
    \end{tabular}
    \end{hyphenrules}
}